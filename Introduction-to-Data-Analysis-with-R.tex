% Options for packages loaded elsewhere
\PassOptionsToPackage{unicode}{hyperref}
\PassOptionsToPackage{hyphens}{url}
\PassOptionsToPackage{dvipsnames,svgnames,x11names}{xcolor}
%
\documentclass[
  letterpaper,
  DIV=11,
  numbers=noendperiod]{scrreprt}

\usepackage{amsmath,amssymb}
\usepackage{iftex}
\ifPDFTeX
  \usepackage[T1]{fontenc}
  \usepackage[utf8]{inputenc}
  \usepackage{textcomp} % provide euro and other symbols
\else % if luatex or xetex
  \usepackage{unicode-math}
  \defaultfontfeatures{Scale=MatchLowercase}
  \defaultfontfeatures[\rmfamily]{Ligatures=TeX,Scale=1}
\fi
\usepackage{lmodern}
\ifPDFTeX\else  
    % xetex/luatex font selection
\fi
% Use upquote if available, for straight quotes in verbatim environments
\IfFileExists{upquote.sty}{\usepackage{upquote}}{}
\IfFileExists{microtype.sty}{% use microtype if available
  \usepackage[]{microtype}
  \UseMicrotypeSet[protrusion]{basicmath} % disable protrusion for tt fonts
}{}
\makeatletter
\@ifundefined{KOMAClassName}{% if non-KOMA class
  \IfFileExists{parskip.sty}{%
    \usepackage{parskip}
  }{% else
    \setlength{\parindent}{0pt}
    \setlength{\parskip}{6pt plus 2pt minus 1pt}}
}{% if KOMA class
  \KOMAoptions{parskip=half}}
\makeatother
\usepackage{xcolor}
\setlength{\emergencystretch}{3em} % prevent overfull lines
\setcounter{secnumdepth}{5}
% Make \paragraph and \subparagraph free-standing
\ifx\paragraph\undefined\else
  \let\oldparagraph\paragraph
  \renewcommand{\paragraph}[1]{\oldparagraph{#1}\mbox{}}
\fi
\ifx\subparagraph\undefined\else
  \let\oldsubparagraph\subparagraph
  \renewcommand{\subparagraph}[1]{\oldsubparagraph{#1}\mbox{}}
\fi

\usepackage{color}
\usepackage{fancyvrb}
\newcommand{\VerbBar}{|}
\newcommand{\VERB}{\Verb[commandchars=\\\{\}]}
\DefineVerbatimEnvironment{Highlighting}{Verbatim}{commandchars=\\\{\}}
% Add ',fontsize=\small' for more characters per line
\usepackage{framed}
\definecolor{shadecolor}{RGB}{241,243,245}
\newenvironment{Shaded}{\begin{snugshade}}{\end{snugshade}}
\newcommand{\AlertTok}[1]{\textcolor[rgb]{0.68,0.00,0.00}{#1}}
\newcommand{\AnnotationTok}[1]{\textcolor[rgb]{0.37,0.37,0.37}{#1}}
\newcommand{\AttributeTok}[1]{\textcolor[rgb]{0.40,0.45,0.13}{#1}}
\newcommand{\BaseNTok}[1]{\textcolor[rgb]{0.68,0.00,0.00}{#1}}
\newcommand{\BuiltInTok}[1]{\textcolor[rgb]{0.00,0.23,0.31}{#1}}
\newcommand{\CharTok}[1]{\textcolor[rgb]{0.13,0.47,0.30}{#1}}
\newcommand{\CommentTok}[1]{\textcolor[rgb]{0.37,0.37,0.37}{#1}}
\newcommand{\CommentVarTok}[1]{\textcolor[rgb]{0.37,0.37,0.37}{\textit{#1}}}
\newcommand{\ConstantTok}[1]{\textcolor[rgb]{0.56,0.35,0.01}{#1}}
\newcommand{\ControlFlowTok}[1]{\textcolor[rgb]{0.00,0.23,0.31}{#1}}
\newcommand{\DataTypeTok}[1]{\textcolor[rgb]{0.68,0.00,0.00}{#1}}
\newcommand{\DecValTok}[1]{\textcolor[rgb]{0.68,0.00,0.00}{#1}}
\newcommand{\DocumentationTok}[1]{\textcolor[rgb]{0.37,0.37,0.37}{\textit{#1}}}
\newcommand{\ErrorTok}[1]{\textcolor[rgb]{0.68,0.00,0.00}{#1}}
\newcommand{\ExtensionTok}[1]{\textcolor[rgb]{0.00,0.23,0.31}{#1}}
\newcommand{\FloatTok}[1]{\textcolor[rgb]{0.68,0.00,0.00}{#1}}
\newcommand{\FunctionTok}[1]{\textcolor[rgb]{0.28,0.35,0.67}{#1}}
\newcommand{\ImportTok}[1]{\textcolor[rgb]{0.00,0.46,0.62}{#1}}
\newcommand{\InformationTok}[1]{\textcolor[rgb]{0.37,0.37,0.37}{#1}}
\newcommand{\KeywordTok}[1]{\textcolor[rgb]{0.00,0.23,0.31}{#1}}
\newcommand{\NormalTok}[1]{\textcolor[rgb]{0.00,0.23,0.31}{#1}}
\newcommand{\OperatorTok}[1]{\textcolor[rgb]{0.37,0.37,0.37}{#1}}
\newcommand{\OtherTok}[1]{\textcolor[rgb]{0.00,0.23,0.31}{#1}}
\newcommand{\PreprocessorTok}[1]{\textcolor[rgb]{0.68,0.00,0.00}{#1}}
\newcommand{\RegionMarkerTok}[1]{\textcolor[rgb]{0.00,0.23,0.31}{#1}}
\newcommand{\SpecialCharTok}[1]{\textcolor[rgb]{0.37,0.37,0.37}{#1}}
\newcommand{\SpecialStringTok}[1]{\textcolor[rgb]{0.13,0.47,0.30}{#1}}
\newcommand{\StringTok}[1]{\textcolor[rgb]{0.13,0.47,0.30}{#1}}
\newcommand{\VariableTok}[1]{\textcolor[rgb]{0.07,0.07,0.07}{#1}}
\newcommand{\VerbatimStringTok}[1]{\textcolor[rgb]{0.13,0.47,0.30}{#1}}
\newcommand{\WarningTok}[1]{\textcolor[rgb]{0.37,0.37,0.37}{\textit{#1}}}

\providecommand{\tightlist}{%
  \setlength{\itemsep}{0pt}\setlength{\parskip}{0pt}}\usepackage{longtable,booktabs,array}
\usepackage{calc} % for calculating minipage widths
% Correct order of tables after \paragraph or \subparagraph
\usepackage{etoolbox}
\makeatletter
\patchcmd\longtable{\par}{\if@noskipsec\mbox{}\fi\par}{}{}
\makeatother
% Allow footnotes in longtable head/foot
\IfFileExists{footnotehyper.sty}{\usepackage{footnotehyper}}{\usepackage{footnote}}
\makesavenoteenv{longtable}
\usepackage{graphicx}
\makeatletter
\def\maxwidth{\ifdim\Gin@nat@width>\linewidth\linewidth\else\Gin@nat@width\fi}
\def\maxheight{\ifdim\Gin@nat@height>\textheight\textheight\else\Gin@nat@height\fi}
\makeatother
% Scale images if necessary, so that they will not overflow the page
% margins by default, and it is still possible to overwrite the defaults
% using explicit options in \includegraphics[width, height, ...]{}
\setkeys{Gin}{width=\maxwidth,height=\maxheight,keepaspectratio}
% Set default figure placement to htbp
\makeatletter
\def\fps@figure{htbp}
\makeatother


% load packages
\usepackage{geometry}
\usepackage{xcolor}
\usepackage{eso-pic}
\usepackage{fancyhdr}
\usepackage{sectsty}
\usepackage{fontspec}
\usepackage{titlesec}


% %% Set page size with a wider right margin
\geometry{a4paper, total={170mm,257mm}}

%% Let's define some colours
\definecolor{light}{HTML}{edf7fa}
\definecolor{highlight}{HTML}{2c6d7d}
\definecolor{dark}{HTML}{336882}

% 
% % Custom command for logo
% \newcommand{\logoinclude}[2][]{%
% \includegraphics[#1]{#2}%
% }
% 
% % Let's add the border on the right-hand side
% \AddToShipoutPicture{%
% \AtPageLowerLeft{%
% \put(\LenToUnit{\dimexpr\paperwidth-1.75cm},0){%
% \color{light}\rule{3cm}{\LenToUnit{\paperheight}}%
% }%
% }%
% % logo
% \AtPageLowerLeft{% start the bar at the bottom right of the page
% \put(\LenToUnit{\dimexpr\paperwidth-1cm},27.2cm){% move it to the top right
% \logoinclude[width=1.2cm]{_extensions/nrennie/PrettyPDF/logo.png}%
% }%
% }%
% }



%% Style the page number
\fancypagestyle{mystyle}{
  % \fancyhf{}
  % \renewcommand\headrulewidth{0pt}
  % Empty style
}

\pagestyle{mystyle}


% Center align chapter titles
\usepackage{titlesec}
\titleformat{\chapter}[display]
  {\normalfont\huge\bfseries\centering\color{dark}}{\chaptertitlename\ \thechapter}{20pt}{\Huge}


% add a border to images
% \let\originalincludegraphics\includegraphics
% \renewcommand{\includegraphics}[2][]{%
%   \fcolorbox{light}{white}{\originalincludegraphics[#1]{#2}}%
% }

\let\originalincludegraphics\includegraphics
\renewcommand{\includegraphics}[2][]{%
  \setlength{\fboxrule}{4pt} % Set border thickness to 2pt
  \fcolorbox{light}{white}{\originalincludegraphics[#1]{#2}}%
}



%% Use some custom fonts
\setsansfont{Ubuntu}[
    Path=_extensions/nrennie/PrettyPDF/Ubuntu/,
    Scale=0.9,
    Extension = .ttf,
    UprightFont=*-Regular,
    BoldFont=*-Bold,
    ItalicFont=*-Italic,
    ]

\setmainfont{Ubuntu}[
    Path=_extensions/nrennie/PrettyPDF/Ubuntu/,
    Scale=0.9,
    Extension = .ttf,
    UprightFont=*-Regular,
    BoldFont=*-Bold,
    ItalicFont=*-Italic,
    ]
\usepackage{fontspec}
\usepackage{multirow}
\usepackage{multicol}
\usepackage{colortbl}
\usepackage{hhline}
\usepackage{longtable}
\usepackage{float}
\usepackage{wrapfig}
\usepackage{array}
\usepackage{hyperref}
\usepackage{hhline}
\newlength\Oldarrayrulewidth
\newlength\Oldtabcolsep
\usepackage{booktabs}
\usepackage{caption}
\KOMAoption{captions}{tableheading}
\makeatletter
\@ifpackageloaded{tcolorbox}{}{\usepackage[skins,breakable]{tcolorbox}}
\@ifpackageloaded{fontawesome5}{}{\usepackage{fontawesome5}}
\definecolor{quarto-callout-color}{HTML}{909090}
\definecolor{quarto-callout-note-color}{HTML}{0758E5}
\definecolor{quarto-callout-important-color}{HTML}{CC1914}
\definecolor{quarto-callout-warning-color}{HTML}{EB9113}
\definecolor{quarto-callout-tip-color}{HTML}{00A047}
\definecolor{quarto-callout-caution-color}{HTML}{FC5300}
\definecolor{quarto-callout-color-frame}{HTML}{acacac}
\definecolor{quarto-callout-note-color-frame}{HTML}{4582ec}
\definecolor{quarto-callout-important-color-frame}{HTML}{d9534f}
\definecolor{quarto-callout-warning-color-frame}{HTML}{f0ad4e}
\definecolor{quarto-callout-tip-color-frame}{HTML}{02b875}
\definecolor{quarto-callout-caution-color-frame}{HTML}{fd7e14}
\makeatother
\makeatletter
\makeatother
\makeatletter
\@ifpackageloaded{bookmark}{}{\usepackage{bookmark}}
\makeatother
\makeatletter
\@ifpackageloaded{caption}{}{\usepackage{caption}}
\AtBeginDocument{%
\ifdefined\contentsname
  \renewcommand*\contentsname{Table of contents}
\else
  \newcommand\contentsname{Table of contents}
\fi
\ifdefined\listfigurename
  \renewcommand*\listfigurename{List of Figures}
\else
  \newcommand\listfigurename{List of Figures}
\fi
\ifdefined\listtablename
  \renewcommand*\listtablename{List of Tables}
\else
  \newcommand\listtablename{List of Tables}
\fi
\ifdefined\figurename
  \renewcommand*\figurename{Figure}
\else
  \newcommand\figurename{Figure}
\fi
\ifdefined\tablename
  \renewcommand*\tablename{Table}
\else
  \newcommand\tablename{Table}
\fi
}
\@ifpackageloaded{float}{}{\usepackage{float}}
\floatstyle{ruled}
\@ifundefined{c@chapter}{\newfloat{codelisting}{h}{lop}}{\newfloat{codelisting}{h}{lop}[chapter]}
\floatname{codelisting}{Listing}
\newcommand*\listoflistings{\listof{codelisting}{List of Listings}}
\makeatother
\makeatletter
\@ifpackageloaded{caption}{}{\usepackage{caption}}
\@ifpackageloaded{subcaption}{}{\usepackage{subcaption}}
\makeatother
\makeatletter
\@ifpackageloaded{tcolorbox}{}{\usepackage[skins,breakable]{tcolorbox}}
\makeatother
\makeatletter
\@ifundefined{shadecolor}{\definecolor{shadecolor}{rgb}{.97, .97, .97}}
\makeatother
\makeatletter
\@ifundefined{codebgcolor}{\definecolor{codebgcolor}{named}{light}}
\makeatother
\makeatletter
\makeatother
\ifLuaTeX
  \usepackage{selnolig}  % disable illegal ligatures
\fi
\IfFileExists{bookmark.sty}{\usepackage{bookmark}}{\usepackage{hyperref}}
\IfFileExists{xurl.sty}{\usepackage{xurl}}{} % add URL line breaks if available
\urlstyle{same} % disable monospaced font for URLs
\hypersetup{
  pdftitle={Introduction to Data Analysis with R},
  pdfauthor={The GRAPH Courses},
  colorlinks=true,
  linkcolor={highlight},
  filecolor={Maroon},
  citecolor={Blue},
  urlcolor={highlight},
  pdfcreator={LaTeX via pandoc}}

\title{Introduction to Data Analysis with R}
\usepackage{etoolbox}
\makeatletter
\providecommand{\subtitle}[1]{% add subtitle to \maketitle
  \apptocmd{\@title}{\par {\large #1 \par}}{}{}
}
\makeatother
\subtitle{(Data with R) With Examples from Public Health and
Epidemiology}
\author{The GRAPH Courses}
\date{}

\begin{document}
\maketitle
\pagestyle{mystyle}

\ifdefined\Shaded\renewenvironment{Shaded}{\begin{tcolorbox}[enhanced, frame hidden, boxrule=0pt, borderline west={3pt}{0pt}{shadecolor}, sharp corners, breakable, colback={codebgcolor}]}{\end{tcolorbox}}\fi

\renewcommand*\contentsname{Table of contents}
{
\hypersetup{linkcolor=}
\setcounter{tocdepth}{2}
\tableofcontents
}
\bookmarksetup{startatroot}

\hypertarget{introduction}{%
\chapter*{Introduction}\label{introduction}}
\addcontentsline{toc}{chapter}{Introduction}

\markboth{Introduction}{Introduction}

This book is a compilation of lesson notes for a structured, multi-week online training program offered by Strategic Insights and Analytics (SI Analytics). To access the accompanying lesson videos, practice datasets, exercises, and additional learning resources, please visit our website at
\href{https://sianalytics.org}{sianalytics.org}.

Strategic Insights and Analytics (SI Analytics) is a professional training and analytics organization focused on building practical data analysis, visualization, and decision-intelligence capacity. Through a combination of instructor-led bootcamps, self-paced courses, and applied project support, SI Analytics equips individuals and organizations with the skills needed to translate data into actionable insight across business, public sector, and global development contexts.


\hypertarget{partners-funders}{%
\section*{Partners \& Funders}\label{partners-funders}}
\addcontentsline{toc}{section}{Partners \& Funders}

\markright{Partners \& Funders}

\begin{itemize}
\tightlist
\item
  Babcock University
\item
  Pacesetters Leadership Center
\item
  VoiceIT Africa
\item
  Grant Masterz
\item
  Neccesitous Exquisite World
\end{itemize}

\bookmarksetup{startatroot}

\hypertarget{setting-up-r-and-rstudio}{%
\chapter{Setting up R and RStudio}\label{setting-up-r-and-rstudio}}

\begin{center}\rule{0.5\linewidth}{0.5pt}\end{center}

\hypertarget{learning-objective}{%
\section*{Learning objective}\label{learning-objective}}

\markright{Learning objective}

\begin{enumerate}
\def\labelenumi{\arabic{enumi}.}
\tightlist
\item
  You can access R and RStudio, either through RStudio.cloud or by
  downloading and installing these software to your computer.
\end{enumerate}

\hypertarget{introduction-1}{%
\section{Introduction}\label{introduction-1}}

To start you off on your R journey, we'll need to set you up with the
required software, R and RStudio. \textbf{R} is the programming language
that you'll use write code, while \textbf{RStudio} is an integrated
development environment (IDE) that makes working with R easier.

\hypertarget{working-locally-vs.-on-the-cloud}{%
\section{Working locally vs.~on the
cloud}\label{working-locally-vs.-on-the-cloud}}

There are two main ways that you can access and work with R and RStudio:
download them to your computer, or use a web server to access them on
the cloud.

Using R and RStudio on the cloud is the less common option, but it may
be the right choice if you are just getting started with programming,
and you do not yet want to worry about installing software. You may also
prefer the cloud option if your local computer is old, slow, or
otherwise unfit for running R.

Below, we go through the setup process for RStudio Cloud, Rstudio on
Windows and RStudio on macOS separately. Jump to the section that is
relevant for you!

\begin{tcolorbox}[enhanced jigsaw, colframe=quarto-callout-caution-color-frame, rightrule=.15mm, opacityback=0, breakable, coltitle=black, colbacktitle=quarto-callout-caution-color!10!white, bottomrule=.15mm, leftrule=.75mm, toprule=.15mm, arc=.35mm, bottomtitle=1mm, colback=white, left=2mm, opacitybacktitle=0.6, titlerule=0mm, title=\textcolor{quarto-callout-caution-color}{\faFire}\hspace{0.5em}{Watch Out}, toptitle=1mm]

RStudio cloud will only give you 25 free project hours per month. After
that, you will need to upgrade to a paid plan. If you think you'll need
more than 25 hours per month, you may want to avoid this option.

\end{tcolorbox}

\hypertarget{rstudio-on-the-cloud}{%
\section{RStudio on the cloud}\label{rstudio-on-the-cloud}}

If you'll be working on the cloud, follow the steps below:

\begin{enumerate}
\def\labelenumi{\arabic{enumi}.}
\item
  Go to the website \href{https://rstudio.cloud}{rstudio.cloud} and
  follow the instructions to sign up for a free account. (We recommend
  signing up with Google if you have a Google account, so you don't need
  to remember any new passwords).
\item
  Once you're done, click on the ``New Project'' icon at the top right,
  and select ``New RStudio Project''.
\end{enumerate}

\includegraphics[width=5.51042in,height=\textheight]{images/new_rstudio_project_cloud.png}

You should see a screen like this:

\includegraphics[width=5.5in,height=\textheight]{images/rstudio_cloud_fresh_window.png}

This is RStudio, your new home for a long time to come!

At the top of the screen, rename the project from ``Untitled Project''
to something like ``r\_intro''.

\includegraphics[width=5.48958in,height=\textheight]{images/rstudio_cloud_name_project.png}

You can start using R by typing code into the ``console'' pane on the
left:

\includegraphics[width=5.4375in,height=\textheight]{images/rstudio_cloud_console.png}

Try using R as a calculator here; type \texttt{2\ +\ 2} and press Enter.

That's it; you're ready to roll. Whenever you want to reopen RStudio,
navigate to rstudio.cloud,

Proceed to the ``wrapping up'' section of the lesson.

\hypertarget{set-up-on-windows}{%
\section{Set up on Windows}\label{set-up-on-windows}}

\hypertarget{download-and-install-r}{%
\subsection{Download and install R}\label{download-and-install-r}}

If you're working on Windows, follow the steps below to download and
install R:

\begin{enumerate}
\def\labelenumi{\arabic{enumi}.}
\item
  Go to \href{https://cran.rstudio.com/}{cran.rstudio.com} to access the
  R installation page. Then click the download link for Windows:

  \includegraphics[width=5.59375in,height=\textheight]{images/cran_select_windows.png}
\item
  Choose the ``base'' sub-directory.

  \includegraphics[width=5.91667in,height=\textheight]{images/install_r_windows_base.png}
\item
  Then click on the download link at the top of the page to download the
  latest version of R:

  \includegraphics[width=5.92708in,height=\textheight]{images/install_r_windows_download2.png}

  Note that the screenshot above may not show the latest version.
\item
  After the download is finished, click on the downloaded file, then
  follow the instructions on the installation pop-up window. During
  installation, you should not have to change any of the defaults; just
  keep clicking ``Next'' until the installation is done.

  Well done! You should now have R on your computer. But you likely
  won't ever need to interact with R directly. Instead you'll use the
  RStudio IDE to work with R. Follow the instructions in the next
  section to get RStudio.
\end{enumerate}

\hypertarget{download-install-run-rstudio}{%
\subsection{Download, install \& run
RStudio}\label{download-install-run-rstudio}}

To download RStudio, go to
\href{https://www.rstudio.com/products/rstudio/download/\#download}{rstudio.com/products/rstudio/download/\#download}
and download the Windows version.

\includegraphics[width=4.16667in,height=\textheight]{images/install_rstudio_button_windows.png}

After the download is finished, click on the downloaded file and follow
the installation instructions.

Once installed, RStudio can be opened like any application on your
computer: press the Windows key to bring up the Start menu, and search
for ``rstudio''. Click to to open the app:

\includegraphics[width=4.16667in,height=\textheight]{images/open_rstudio_windows.png}

You should see a window like this:

\includegraphics[width=6.1875in,height=\textheight]{images/rstudio_first_view_windows.png}

This is RStudio, your new home for a long time to come!

You can start using R by typing code into the ``console'' pane on the
left:

\includegraphics[width=5.20833in,height=\textheight]{images/windows_write_code_here.png}

Try using R as a calculator here; type \texttt{2\ +\ 2} and press Enter.

That's it; you're ready to roll. Proceed to the ``wrapping up'' section
of the lesson.

\hypertarget{set-up-on-macos}{%
\section{Set up on macOS}\label{set-up-on-macos}}

\hypertarget{download-and-install-r-1}{%
\subsection{Download and install R}\label{download-and-install-r-1}}

If you're working on macOS, follow the steps below to download and
install R:

\begin{enumerate}
\def\labelenumi{\arabic{enumi}.}
\item
  Go to \href{https://cran.rstudio.com/}{cran.rstudio.com} to access the
  R installation page. Then click the link for macOS:

  \includegraphics[width=4.46875in,height=\textheight]{images/install_r_select_mac.png}
\item
  Download and install the relevant R version for your Mac. For most
  people, the first option under ``Latest release'' will be the one to
  get.

  \includegraphics[width=4.39583in,height=\textheight]{images/r_mac_select_version.png}
\item
  After the download is finished, click on the downloaded file, then
  follow the instructions on the installation pop-up window.
\end{enumerate}

Well done! You should now have R on your computer. But you likely won't
ever need to interact with R directly. Instead you'll use the RStudio
IDE to work with R. Follow the instructions in the next section to get
RStudio.

\hypertarget{download-install-run-rstudio-1}{%
\subsection{Download, install \& run
RStudio}\label{download-install-run-rstudio-1}}

To download RStudio, go to
\href{https://www.rstudio.com/products/rstudio/download/\#download}{rstudio.com/products/rstudio/download/\#download}
and download the version for macOS.

\includegraphics[width=4.16667in,height=\textheight]{images/download_rstudio_button_mac.png}

After the download is finished, click on the downloaded file and follow
the installation instructions.

Once installed, RStudio can be opened like any application on your
computer: Press \texttt{Command} + \texttt{Space} to open Spotlight,
then search for ``rstudio''. Click to open the app.

\includegraphics[width=4.16667in,height=\textheight]{images/mac_spotlight_rstudio.png}

You should see a window like this:

\includegraphics[width=5.375in,height=\textheight]{images/mac_rstudio_window.png}

This is RStudio, your new home for a long time to come!

You can start using R by typing code into the ``console'' pane on the
left:

\includegraphics[width=5.20833in,height=\textheight]{images/mac_write_code_here.png}

Try using R as a calculator here; type \texttt{2\ +\ 2} and press Enter.

\hypertarget{wrap-up}{%
\section{Wrap up}\label{wrap-up}}

You should now have access to R and RStudio, so you're all set to begin
the journey of learning to use these immensely powerful tools. See you
in the next session!

\hypertarget{references}{%
\section*{References}\label{references}}

\markright{References}

Some material in this lesson was adapted from the following sources:

\begin{itemize}
\tightlist
\item
  Nordmann, Emily, and Heather Cleland-Woods. \emph{Chapter 2
  Programming Basics \textbar{} Data Skills}.
  \emph{psyteachr.github.io},
  \url{https://psyteachr.github.io/data-skills-v1/programming-basics.html}
  Accessed 23 Feb.~2022.
\end{itemize}

\bookmarksetup{startatroot}

\hypertarget{setting-up-r-and-rstudio-1}{%
\chapter{Setting up R and RStudio}\label{setting-up-r-and-rstudio-1}}

\begin{center}\rule{0.5\linewidth}{0.5pt}\end{center}

\hypertarget{learning-objective-1}{%
\section*{Learning objective}\label{learning-objective-1}}

\markright{Learning objective}

\begin{enumerate}
\def\labelenumi{\arabic{enumi}.}
\tightlist
\item
  You can access R and RStudio, either through RStudio.cloud or by
  downloading and installing these software to your computer.
\end{enumerate}

\hypertarget{introduction-2}{%
\section{Introduction}\label{introduction-2}}

To start you off on your R journey, we'll need to set you up with the
required software, R and RStudio. \textbf{R} is the programming language
that you'll use write code, while \textbf{RStudio} is an integrated
development environment (IDE) that makes working with R easier.

\hypertarget{working-locally-vs.-on-the-cloud-1}{%
\section{Working locally vs.~on the
cloud}\label{working-locally-vs.-on-the-cloud-1}}

There are two main ways that you can access and work with R and RStudio:
download them to your computer, or use a web server to access them on
the cloud.

Using R and RStudio on the cloud is the less common option, but it may
be the right choice if you are just getting started with programming,
and you do not yet want to worry about installing software. You may also
prefer the cloud option if your local computer is old, slow, or
otherwise unfit for running R.

Below, we go through the setup process for RStudio Cloud, Rstudio on
Windows and RStudio on macOS separately. Jump to the section that is
relevant for you!

\begin{tcolorbox}[enhanced jigsaw, colframe=quarto-callout-caution-color-frame, rightrule=.15mm, opacityback=0, breakable, coltitle=black, colbacktitle=quarto-callout-caution-color!10!white, bottomrule=.15mm, leftrule=.75mm, toprule=.15mm, arc=.35mm, bottomtitle=1mm, colback=white, left=2mm, opacitybacktitle=0.6, titlerule=0mm, title=\textcolor{quarto-callout-caution-color}{\faFire}\hspace{0.5em}{Watch Out}, toptitle=1mm]

RStudio cloud will only give you 25 free project hours per month. After
that, you will need to upgrade to a paid plan. If you think you'll need
more than 25 hours per month, you may want to avoid this option.

\end{tcolorbox}

\hypertarget{rstudio-on-the-cloud-1}{%
\section{RStudio on the cloud}\label{rstudio-on-the-cloud-1}}

If you'll be working on the cloud, follow the steps below:

\begin{enumerate}
\def\labelenumi{\arabic{enumi}.}
\item
  Go to the website \href{https://rstudio.cloud}{rstudio.cloud} and
  follow the instructions to sign up for a free account. (We recommend
  signing up with Google if you have a Google account, so you don't need
  to remember any new passwords).
\item
  Once you're done, click on the ``New Project'' icon at the top right,
  and select ``New RStudio Project''.
\end{enumerate}

\includegraphics[width=5.51042in,height=\textheight]{images/new_rstudio_project_cloud.png}

You should see a screen like this:

\includegraphics[width=5.5in,height=\textheight]{images/rstudio_cloud_fresh_window.png}

This is RStudio, your new home for a long time to come!

At the top of the screen, rename the project from ``Untitled Project''
to something like ``r\_intro''.

\includegraphics[width=5.48958in,height=\textheight]{images/rstudio_cloud_name_project.png}

You can start using R by typing code into the ``console'' pane on the
left:

\includegraphics[width=5.4375in,height=\textheight]{images/rstudio_cloud_console.png}

Try using R as a calculator here; type \texttt{2\ +\ 2} and press Enter.

That's it; you're ready to roll. Whenever you want to reopen RStudio,
navigate to rstudio.cloud,

Proceed to the ``wrapping up'' section of the lesson.

\hypertarget{set-up-on-windows-1}{%
\section{Set up on Windows}\label{set-up-on-windows-1}}

\hypertarget{download-and-install-r-2}{%
\subsection{Download and install R}\label{download-and-install-r-2}}

If you're working on Windows, follow the steps below to download and
install R:

\begin{enumerate}
\def\labelenumi{\arabic{enumi}.}
\item
  Go to \href{https://cran.rstudio.com/}{cran.rstudio.com} to access the
  R installation page. Then click the download link for Windows:

  \includegraphics[width=5.59375in,height=\textheight]{images/cran_select_windows.png}
\item
  Choose the ``base'' sub-directory.

  \includegraphics[width=5.91667in,height=\textheight]{images/install_r_windows_base.png}
\item
  Then click on the download link at the top of the page to download the
  latest version of R:

  \includegraphics[width=5.92708in,height=\textheight]{images/install_r_windows_download2.png}

  Note that the screenshot above may not show the latest version.
\item
  After the download is finished, click on the downloaded file, then
  follow the instructions on the installation pop-up window. During
  installation, you should not have to change any of the defaults; just
  keep clicking ``Next'' until the installation is done.

  Well done! You should now have R on your computer. But you likely
  won't ever need to interact with R directly. Instead you'll use the
  RStudio IDE to work with R. Follow the instructions in the next
  section to get RStudio.
\end{enumerate}

\hypertarget{download-install-run-rstudio-2}{%
\subsection{Download, install \& run
RStudio}\label{download-install-run-rstudio-2}}

To download RStudio, go to
\href{https://www.rstudio.com/products/rstudio/download/\#download}{rstudio.com/products/rstudio/download/\#download}
and download the Windows version.

\includegraphics[width=4.16667in,height=\textheight]{images/install_rstudio_button_windows.png}

After the download is finished, click on the downloaded file and follow
the installation instructions.

Once installed, RStudio can be opened like any application on your
computer: press the Windows key to bring up the Start menu, and search
for ``rstudio''. Click to to open the app:

\includegraphics[width=4.16667in,height=\textheight]{images/open_rstudio_windows.png}

You should see a window like this:

\includegraphics[width=6.1875in,height=\textheight]{images/rstudio_first_view_windows.png}

This is RStudio, your new home for a long time to come!

You can start using R by typing code into the ``console'' pane on the
left:

\includegraphics[width=5.20833in,height=\textheight]{images/windows_write_code_here.png}

Try using R as a calculator here; type \texttt{2\ +\ 2} and press Enter.

That's it; you're ready to roll. Proceed to the ``wrapping up'' section
of the lesson.

\hypertarget{set-up-on-macos-1}{%
\section{Set up on macOS}\label{set-up-on-macos-1}}

\hypertarget{download-and-install-r-3}{%
\subsection{Download and install R}\label{download-and-install-r-3}}

If you're working on macOS, follow the steps below to download and
install R:

\begin{enumerate}
\def\labelenumi{\arabic{enumi}.}
\item
  Go to \href{https://cran.rstudio.com/}{cran.rstudio.com} to access the
  R installation page. Then click the link for macOS:

  \includegraphics[width=4.46875in,height=\textheight]{images/install_r_select_mac.png}
\item
  Download and install the relevant R version for your Mac. For most
  people, the first option under ``Latest release'' will be the one to
  get.

  \includegraphics[width=4.39583in,height=\textheight]{images/r_mac_select_version.png}
\item
  After the download is finished, click on the downloaded file, then
  follow the instructions on the installation pop-up window.
\end{enumerate}

Well done! You should now have R on your computer. But you likely won't
ever need to interact with R directly. Instead you'll use the RStudio
IDE to work with R. Follow the instructions in the next section to get
RStudio.

\hypertarget{download-install-run-rstudio-3}{%
\subsection{Download, install \& run
RStudio}\label{download-install-run-rstudio-3}}

To download RStudio, go to
\href{https://www.rstudio.com/products/rstudio/download/\#download}{rstudio.com/products/rstudio/download/\#download}
and download the version for macOS.

\includegraphics[width=4.16667in,height=\textheight]{images/download_rstudio_button_mac.png}

After the download is finished, click on the downloaded file and follow
the installation instructions.

Once installed, RStudio can be opened like any application on your
computer: Press \texttt{Command} + \texttt{Space} to open Spotlight,
then search for ``rstudio''. Click to open the app.

\includegraphics[width=4.16667in,height=\textheight]{images/mac_spotlight_rstudio.png}

You should see a window like this:

\includegraphics[width=5.375in,height=\textheight]{images/mac_rstudio_window.png}

This is RStudio, your new home for a long time to come!

You can start using R by typing code into the ``console'' pane on the
left:

\includegraphics[width=5.20833in,height=\textheight]{images/mac_write_code_here.png}

Try using R as a calculator here; type \texttt{2\ +\ 2} and press Enter.

\hypertarget{wrap-up-1}{%
\section{Wrap up}\label{wrap-up-1}}

You should now have access to R and RStudio, so you're all set to begin
the journey of learning to use these immensely powerful tools. See you
in the next session!

\hypertarget{references-1}{%
\section*{References}\label{references-1}}

\markright{References}

Some material in this lesson was adapted from the following sources:

\begin{itemize}
\tightlist
\item
  Nordmann, Emily, and Heather Cleland-Woods. \emph{Chapter 2
  Programming Basics \textbar{} Data Skills}.
  \emph{psyteachr.github.io},
  \url{https://psyteachr.github.io/data-skills-v1/programming-basics.html}
  Accessed 23 Feb.~2022.
\end{itemize}

\bookmarksetup{startatroot}

\hypertarget{using-rstudio}{%
\chapter{Using RStudio}\label{using-rstudio}}

\begin{center}\rule{0.5\linewidth}{0.5pt}\end{center}

\hypertarget{learning-objectives}{%
\section{Learning objectives}\label{learning-objectives}}

\begin{enumerate}
\def\labelenumi{\arabic{enumi}.}
\item
  You can identify and use the following tabs in RStudio: Source,
  Console, Environment, History, Files, Plots, Packages, Help and
  Viewer.
\item
  You can modify RStudio's interface options to suit your needs.
\end{enumerate}

\hypertarget{introduction-3}{%
\section{Introduction}\label{introduction-3}}

Now that you have access to R \& RStudio, let's go on a quick tour of
the RStudio interface, your digital home for a long time to come.

We will cover a lot of territory quickly. Do not panic. You are not
expected to remember it all this. Rather, you will see these topics
again and again throughout the course, and you will naturally assimilate
them that way.

You can also refer back to this lesson as you progress.

The goal here is simply to make you aware of the tools at your disposal
within RStudio.

\begin{center}\rule{0.5\linewidth}{0.5pt}\end{center}

To get started, you need to open the RStudio application:

\begin{itemize}
\item
  If you are working with RStudio Cloud, go to
  \href{https://rstudio.cloud}{rstudio.cloud}, log in, then click on the
  ``r\_intro'' project that you created in the last lesson. (If you do
  not see this, simply create a new R project using the ``New Project''
  icon at the top right).
\item
  If you are working on your local computer, go to your applications
  folder and double click on the RStudio icon. Or you search for this
  application from your Start Menu (Windows), or through Spotlight
  (Mac).
\end{itemize}

\hypertarget{the-rstudio-panes}{%
\section{The RStudio panes}\label{the-rstudio-panes}}

By default, RStudio is arranged into four window panes.

If you only see three panes, open a new script with
\texttt{File\ \textgreater{}\ New\ File\ \textgreater{}\ R\ Script} .
This should reveal one more pane.

\includegraphics[width=4.69792in,height=0.73958in]{images/new_r_script.jpg}

Before we go any further, we will rearrange these panes to improve the
usability of the interface.

To do this, in the RStudio menu at the top of the screen, select
\texttt{Tools\ \textgreater{}\ Global\ Options} to bring up RStudio's
options. Then under \texttt{Pane\ Layout}, adjust the pane arrangement.
The arrangement we recommend is shown below.

\includegraphics[width=4.19792in,height=\textheight]{images/rstudio_pane_layout.png}

At the top left pane is the Source tab, and at the top right pane, you
should have the Console tab.

Then at the bottom left pane, no tab options should checked---this
section should be left empty, with the drop-down saying just ``TabSet''.

Finally, at the bottom right pane, you should check the following tabs:
Environment, History, Files, Plots, Packages, Help and Viewer.

\begin{center}\rule{0.5\linewidth}{0.5pt}\end{center}

Great, now you should have an RStudio window that looks something like
this:

\includegraphics[width=5.08333in,height=\textheight]{images/rstudio_four_panes_rearranged.jpg}

\begin{center}\rule{0.5\linewidth}{0.5pt}\end{center}

The top-left pane is where you will do most of the coding. Make this
larger by clicking on its maximize icon:

\includegraphics[width=3.67708in,height=\textheight]{images/maximize_icon.jpg}

\begin{center}\rule{0.5\linewidth}{0.5pt}\end{center}

Note that you can drag the bar that separates the window panes to resize
them.

\includegraphics[width=2.40625in,height=\textheight]{images/drag_to_resize_panes.png}

\begin{center}\rule{0.5\linewidth}{0.5pt}\end{center}

Now let's look at each of the RStudio tabs one by one. Below is a
summary image of what we will discuss:

\includegraphics{images/rstudio_panes_detail.jpg}

\hypertarget{sourceeditor}{%
\subsection{Source/Editor}\label{sourceeditor}}

\includegraphics[width=4.88542in,height=\textheight]{images/rstudio_editor.png}

The source or editor is where your R ``scripts'' go. A script is a text
document where you write and save code.

Because this is where you will do most of your coding, it is important
that you have a lot of visual space. That is why we rearranged the
RStudio pane layout above---to give the Editor more space.

Now let's see how to use this Editor.

\begin{center}\rule{0.5\linewidth}{0.5pt}\end{center}

First, \textbf{open a new script} under the File menu if one is not yet
open:
\texttt{File\ \textgreater{}\ New\ File\ \textgreater{}\ R\ Script}. In
the script, type the following:

\begin{Shaded}
\begin{Highlighting}[]
\FunctionTok{print}\NormalTok{(}\StringTok{"excited for R!"}\NormalTok{)}
\end{Highlighting}
\end{Shaded}

To \textbf{run code}, place your cursor anywhere in the code, then hit
\texttt{Command} + \texttt{Enter} on macOS, or \texttt{Control} +
\texttt{Enter} on Windows.

This should send the code to the Console and run it.

\begin{center}\rule{0.5\linewidth}{0.5pt}\end{center}

You can also \textbf{run multiple lines at once}. To try this, add a
second line to your script, so that it now reads:

\begin{Shaded}
\begin{Highlighting}[]
\FunctionTok{print}\NormalTok{(}\StringTok{"excited for R!"}\NormalTok{)}
\FunctionTok{print}\NormalTok{(}\StringTok{"and RStudio!"}\NormalTok{)}
\end{Highlighting}
\end{Shaded}

Now drag your cursor to highlight both lines and press
\texttt{Command}/\texttt{Control} + \texttt{Enter}.

To \textbf{run the entire script}, you can use
\texttt{Command}/\texttt{Control} + \texttt{A} to select all code, then
press \texttt{Command}/\texttt{Control} + \texttt{Enter}. Try this now.
Deselect your code, then try to the shortcut to select all.

\begin{tcolorbox}[enhanced jigsaw, colframe=quarto-callout-note-color-frame, rightrule=.15mm, opacityback=0, breakable, coltitle=black, colbacktitle=quarto-callout-note-color!10!white, bottomrule=.15mm, leftrule=.75mm, toprule=.15mm, arc=.35mm, bottomtitle=1mm, colback=white, left=2mm, opacitybacktitle=0.6, titlerule=0mm, title=\textcolor{quarto-callout-note-color}{\faInfo}\hspace{0.5em}{Side Note}, toptitle=1mm]

There is also a `Run' button at the top right of the source panel (
\includegraphics[width=0.38542in,height=\textheight]{images/run_icon.jpg}
), with which you can run code (either the current line, or all
highlighted code). But you should try to use the keyboard shortcut
instead.

\end{tcolorbox}

\begin{center}\rule{0.5\linewidth}{0.5pt}\end{center}

To \textbf{open the script in a new window}, click on the third icon in
the toolbar directly above the script.

\includegraphics[width=3.42708in,height=\textheight]{images/rstudio_pop_out_icon.png}

To put the window back, click on the same button on the now-external
window.

\begin{center}\rule{0.5\linewidth}{0.5pt}\end{center}

Next, \textbf{save the script.} Hit \texttt{Command}/\texttt{Control} +
\texttt{S} to bring up the Save dialog box. Give it a file name like
``rstudio\_intro''.

\begin{itemize}
\item
  If you are working with RStudio cloud, the file will be saved in your
  project folder.
\item
  If you are working on your local computer, save the file in an
  easy-to-locate part of your computer, perhaps your desktop. (Later on
  we will think about the ``proper'' way to organize and store scripts).
\end{itemize}

\begin{center}\rule{0.5\linewidth}{0.5pt}\end{center}

You can \textbf{view data frames} (which are like spreadsheets in R) in
the same pane. To observe this, type and run the code below on a new
line in your script:

\begin{Shaded}
\begin{Highlighting}[]
\FunctionTok{View}\NormalTok{(women)}
\end{Highlighting}
\end{Shaded}

Notice the uppercase ``V'' in \texttt{View()}.

\includegraphics[width=2.4375in,height=\textheight]{images/rstudio_view_data_frame.png}

\texttt{women} is the name of a dataset that comes loaded with R. It
gives the average heights and weights for American women aged 30--39.

You can click on the ``x'' icon to the right of the ``women'' tab to
close this data viewer.

\begin{center}\rule{0.5\linewidth}{0.5pt}\end{center}

\hypertarget{console}{%
\subsection{Console}\label{console}}

The \emph{console}, at the bottom left, is where \textbf{code is
executed}. You can type code directly here, but it will not be saved.

Type a random piece of code (maybe a calculation like \texttt{3\ +\ 3})
and press `Enter'.

\includegraphics[width=6.1875in,height=\textheight]{images/rstudio_console.png}

If you place your cursor on the last line of the console, and you press
the \textbf{up arrow}, you can go back to the last code that was run.
Keep pressing it to cycle to the previous lines.

To run any of these previous lines, press \emph{Enter}.

\hypertarget{environment}{%
\subsection{Environment}\label{environment}}

\includegraphics[width=5.48958in,height=\textheight]{images/environment_tab.png}

At the top right of the RStudio Window, you should see the
\textbf{Environment} tab.

The Environment tab shows datasets and other objects that are loaded
into R's working memory, or ``workspace''.

To explore this tab, let's import a dataset into your environment from
the web. Type the code below into your script and run it:

\begin{Shaded}
\begin{Highlighting}[]
\NormalTok{ebola\_data }\OtherTok{\textless{}{-}} \FunctionTok{read.csv}\NormalTok{(}\StringTok{"https://tinyurl.com/ebola{-}data{-}sample"}\NormalTok{)}
\end{Highlighting}
\end{Shaded}

\begin{tcolorbox}[enhanced jigsaw, colframe=quarto-callout-note-color-frame, rightrule=.15mm, opacityback=0, breakable, coltitle=black, colbacktitle=quarto-callout-note-color!10!white, bottomrule=.15mm, leftrule=.75mm, toprule=.15mm, arc=.35mm, bottomtitle=1mm, colback=white, left=2mm, opacitybacktitle=0.6, titlerule=0mm, title=\textcolor{quarto-callout-note-color}{\faInfo}\hspace{0.5em}{Side Note}, toptitle=1mm]

You don't need to understand exactly what the code above is doing for
now. We just want to quickly show you the basic features of the
Environment pane; we'll look at data importing in detail later.

Also, if you do not have active internet access, the code above will not
run. You can skip this section and move to the ``History'' tab.

\end{tcolorbox}

You have now imported the dataset and stored it in an \emph{object}
named \texttt{ebola\_data}. (You could have named the object anything
you want.)

Now that the dataset is stored by R, you should be able to see it in the
Environment pane. If you click on the blue drop-down icon beside the
object's name in the Environment tab to reveal a summary.

\includegraphics[width=3.65625in,height=\textheight]{images/rstudio_explore_object_dropdown.png}

Try clicking directly on the \texttt{ebola\_data} dataset from the
Environment tab. This opens it in a `View' tab.

\begin{center}\rule{0.5\linewidth}{0.5pt}\end{center}

You can \textbf{remove an object from the workspace} with the
\texttt{rm()} function. Type and run the following in a new line on your
R script.

\begin{Shaded}
\begin{Highlighting}[]
\FunctionTok{rm}\NormalTok{(ebola\_data)}
\end{Highlighting}
\end{Shaded}

Notice that the \texttt{ebola\_data} object no longer shows up in your
environment after having run that code.

The broom icon, at the top of the Environment pane can also be used to
clear your workspace.

\includegraphics{images/broom_icon.png}

To practice using it, try re-running the line above that imports the
Ebola dataset, then clear the object using the broom icon.

\hypertarget{history}{%
\subsection{History}\label{history}}

Next, the \textbf{History} tab shows previous commands you have run.

\includegraphics[width=4.89583in,height=\textheight]{images/history_tab.png}

You can click a line to highlight it, then send it to the console or to
your script with the ``To Console'' and ``To Source'' icons at the top
of this tab.

To select multiple lines, use the ``Shift-click'' method: click the
first item you want to select, then hold down the ``Shift'' key and
click the last item you want to select.

Finally, notice that there is a search bar at the top right of the
History pane where you can search for past commands that you have run.

\hypertarget{files}{%
\subsection{Files}\label{files}}

Next, the \textbf{Files} tab. This shows the files and folders in the
folder you are working in.

\includegraphics{images/rstudio_files.png}

The tab allows you to interact with your computer's file system.

Try playing with some of the buttons here, to see what they do. You
should try at least the following:

\begin{itemize}
\item
  Make a new folder
\item
  Delete that folder
\item
  Make a new R Script
\item
  Rename that script
\end{itemize}

\hypertarget{plots}{%
\subsection{Plots}\label{plots}}

Next, the \textbf{Plots} tab. This is where figures that are generated
by R will show up. Try creating a simple plot with the following code:

\begin{Shaded}
\begin{Highlighting}[]
\FunctionTok{plot}\NormalTok{(women)}
\end{Highlighting}
\end{Shaded}

\includegraphics{images/rstudio_plots.png}

That code creates a plot of the two variables in the \texttt{women}
dataset. You should see this figure in the Plots tab.

Now, test out the buttons at the top of this tab to explore what they
do. In particular, try to export a plot to your computer.

\hypertarget{packages}{%
\subsection{Packages}\label{packages}}

Next, let's look at the \textbf{Packages} tab.

\includegraphics[width=4.36458in,height=\textheight]{images/packages_tab.png}

Packages are collections of R code that extend the functionality of R.
We will discuss packages in detail in a future lesson.

For now, it is important to know that to use a package, you need to
\emph{install} then \emph{load} it. Packages need to be installed only
once, but must be loaded in each new R session.

All the package names you see (in blue font) are packages that are
installed on your system. And packages with a checkmark are packages
which are \emph{loaded} in the current session.

You can install a package with the Install button of the Packages tab.

\includegraphics[width=3.625in,height=\textheight]{images/rstudio_install_package_icon.png}

But it is better to install and load packages with R code, rather than
the Install button. Let's try this. Type and run the code below to
install the \{highcharter\} package.

\begin{Shaded}
\begin{Highlighting}[]
\FunctionTok{install.packages}\NormalTok{(}\StringTok{"highcharter"}\NormalTok{)}
\FunctionTok{library}\NormalTok{(highcharter)}
\end{Highlighting}
\end{Shaded}

The first line installs the package. The second line \emph{loads} the
package from your package library.

Because you only need to install a package once, you can now remove the
installation line from your script.

\begin{center}\rule{0.5\linewidth}{0.5pt}\end{center}

Now that the \{highcharter\} package has been installed and loaded, you
can use the functions that come in the package. To try this, type and
run the code below:

\begin{Shaded}
\begin{Highlighting}[]
\NormalTok{highcharter}\SpecialCharTok{::}\FunctionTok{hchart}\NormalTok{(women}\SpecialCharTok{$}\NormalTok{weight)}
\end{Highlighting}
\end{Shaded}

\begin{verbatim}
Registered S3 method overwritten by 'quantmod':
  method            from
  as.zoo.data.frame zoo 
\end{verbatim}

\begin{figure}[H]

{\centering \includegraphics{foundations_ls02_using_rstudio_files/figure-pdf/unnamed-chunk-9-1.pdf}

}

\end{figure}

This code uses the \texttt{hchart()} \emph{function} from the
\{highcharter\} package to plot an interactive histogram showing the
distribution of weights in the \texttt{women} dataset.

(Of course, you may not yet know what a function is. We'll get to this
soon.)

\hypertarget{viewer}{%
\subsection{Viewer}\label{viewer}}

Notice that the histogram above shows up in a \textbf{Viewer} tab. This
tab allows you to preview HTML files and interactive objects.

\hypertarget{help}{%
\subsection{Help}\label{help}}

Lastly, the \textbf{Help} tab shows the documentation for different R
objects. Try typing out and running each line below to see what this
documentation looks like.

\begin{Shaded}
\begin{Highlighting}[]
\NormalTok{?hchart}
\NormalTok{?women}
\NormalTok{?read.csv}
\end{Highlighting}
\end{Shaded}

\includegraphics[width=4.69792in,height=\textheight]{images/example_help_page.png}

Help files are not always very easy to understand for beginners, but
with time they will become more useful.

\hypertarget{rstudio-options}{%
\section{RStudio options}\label{rstudio-options}}

RStudio has a number of useful options for changing it's look and
functionality. Let's try these. You may not understand all the changes
made for now. That's fine.

In the RStudio menu at the top of the screen, select
\texttt{Tools\ \textgreater{}\ Global\ Options} to bring up RStudio's
options.

\begin{itemize}
\item
  Now, under \texttt{Appearance}, choose your ideal theme. (We like the
  ``Crimson Editor'' and ``Tomorrow Night'' themes.)

  \includegraphics[width=5.52083in,height=\textheight]{images/rstudio_themes.png}
\item
  Under \texttt{Code\ \textgreater{}\ Display}, check ``Highlight R
  function calls''. What this does is give your R \emph{functions} a
  unique color, improving readability. You will understand this later.
\item
  Also under \texttt{Code\ \textgreater{}\ Display}, check ``Rainbow
  parentheses''. What this does is make your ``nested parentheses''
  easier to read by giving each pair a unique color.

  \includegraphics[width=4.64583in,height=\textheight]{images/options_highlight_function_rainbow_parentheses.jpg}

  \includegraphics[width=4.79167in,height=0.6875in]{images/highlight_r_function_calls.jpg}

  \includegraphics[width=3.76042in,height=\textheight]{images/rainbow_parentheses.jpg}
\item
  Finally under \texttt{General\ \textgreater{}\ Basic},
  \textbf{uncheck} the box that says \textbf{``Restore .RData into
  workspace at startup''}. You don't want to restore any data to your
  workspace (or \emph{environment)} when you start RStudio. Starting
  with a clean workspace each time is less likely to lead to errors.

  This also means that you never want to \textbf{``save your workspace
  to .RData on exit''}, so set this to \textbf{Never}.
\end{itemize}

\hypertarget{command-palette}{%
\section{Command palette}\label{command-palette}}

The Rstudio command palette gives instant, searchable access to many of
the RStudio menu options and settings that we have seen so far.

The palette can be invoked with the keyboard shortcut \texttt{Ctrl} +
\texttt{Shift} + \texttt{P} (\texttt{Cmd} + \texttt{Shift} + \texttt{P}
on macOS).

It's also available on the \emph{Tools} menu (\emph{Tools}
-\textgreater{} \emph{Show Command Palette}).

\includegraphics[width=5.48958in,height=\textheight]{images/command_palette.jpg}

Try using it to:

\begin{itemize}
\item
  Create a new script (Search ``new script'' and click on the relevant
  option)
\item
  Rename a script (Search ``rename'' and click on the relevant option)
\end{itemize}

\hypertarget{wrapping-up}{%
\section{Wrapping up}\label{wrapping-up}}

Congratulations! You are now a new citizen of RStudio.

Of course, you have only scratched the surface of RStudio functionality.
As you advance in your R journey, you will discover new features, and
you will hopefully grow to love the wonderful integrated development
environment (IDE) that is RStudio. One good place to start is the
official RStudio IDE
\href{https://thegraphcourses.org/wp-content/uploads/2022/03/rstudio-IDE-cheatsheet.pdf}{cheatsheet}.

Below is one section of that sheet:

\includegraphics{images/rstudio_cheatsheet.png}

See you in the next lesson!

\hypertarget{further-resources}{%
\section{Further resources}\label{further-resources}}

\begin{enumerate}
\def\labelenumi{\arabic{enumi}.}
\tightlist
\item
  \href{https://www.dataquest.io/blog/rstudio-tips-tricks-shortcuts/}{23
  RStudio Tips, Tricks, and Shortcuts}
\end{enumerate}

\hypertarget{references-2}{%
\section{References}\label{references-2}}

Some material in this lesson was adapted from the following sources:

\begin{itemize}
\tightlist
\item
  ``Rstudio Cheatsheets.'' \emph{RStudio},
  \url{https://www.rstudio.com/resources/cheatsheets/.}
\item
  ``Chapter 1 Getting Started: Data Skills for Reproducible Research.''
  \emph{Chapter 1 Getting Started \textbar{} Data Skills for
  Reproducible Research},
  \url{https://psyteachr.github.io/reprores-v2/intro.html.}
\end{itemize}

\bookmarksetup{startatroot}

\hypertarget{coding-basics}{%
\chapter{Coding basics}\label{coding-basics}}

\begin{center}\rule{0.5\linewidth}{0.5pt}\end{center}

\hypertarget{learning-objectives-1}{%
\section*{Learning objectives}\label{learning-objectives-1}}

\markright{Learning objectives}

\begin{enumerate}
\def\labelenumi{\arabic{enumi}.}
\item
  You can write comments in R.
\item
  You can create section headers in RStudio.
\item
  You know how to use R as a calculator.
\item
  You can create, overwrite and manipulate R objects.
\item
  You understand the basic rules for naming R objects.
\item
  You understand the syntax for calling R functions.
\item
  You know how to nest multiple functions.
\item
  You can use install and load add-on R packages and call functions from
  these packages.
\end{enumerate}

\hypertarget{introduction-4}{%
\section{Introduction}\label{introduction-4}}

In the last lesson, you learned how to use RStudio, the wonderful
integrated development environment (IDE) that makes working with R much
easier. In this lesson, you will learn the basics of using R itself.

To get started, open RStudio, and open a new script with
\texttt{File\ \textgreater{}\ New\ File\ \textgreater{}\ R\ Script} on
the RStudio menu.

\includegraphics[width=4.69792in,height=0.73958in]{images/new_r_script.jpg}

Next, \textbf{save the script} with \texttt{File\ \textgreater{}\ Save}
on the RStudio menu or by using the shortcut
\texttt{Command}/\texttt{Control} + \texttt{S} . This should bring up
the Save File dialog box. Save the file with a name like
``coding\_basics''.

You should now type all the code from this lesson into that script.

\hypertarget{comments}{%
\section{Comments}\label{comments}}

There are two main types of text in an R script: commands and comments.
A command is a line or lines of R code that instructs R to do something
(e.g.~\texttt{2\ +\ 2})

A comment is text that is ignored by the computer.

Anything that follows a \texttt{\#} symbol (pronounced ``hash'' or
``pound'') on a given line is a comment. Try typing out and running the
code below to see this:

\begin{Shaded}
\begin{Highlighting}[]
\DocumentationTok{\#\# A comment}
\DecValTok{2} \SpecialCharTok{+} \DecValTok{2} \CommentTok{\# Another comment}
\DocumentationTok{\#\# 2 + 2}
\end{Highlighting}
\end{Shaded}

Since they are ignored by the computer, comments are meant for
\emph{humans.} They help you and others keep track of what your code is
doing. Use them often! Like your mother always says, ``too much
everything is bad, except for R comments''.

\begin{tcolorbox}[enhanced jigsaw, colframe=quarto-callout-tip-color-frame, rightrule=.15mm, opacityback=0, breakable, coltitle=black, colbacktitle=quarto-callout-tip-color!10!white, bottomrule=.15mm, leftrule=.75mm, toprule=.15mm, arc=.35mm, bottomtitle=1mm, colback=white, left=2mm, opacitybacktitle=0.6, titlerule=0mm, title=\textcolor{quarto-callout-tip-color}{\faLightbulb}\hspace{0.5em}{Practice}, toptitle=1mm]

\textbf{Question 1}

True or False: both code chunks below are valid ways to comment code:?

\begin{Shaded}
\begin{Highlighting}[]
\DocumentationTok{\#\# add two numbers}
\DecValTok{2} \SpecialCharTok{+} \DecValTok{2}
\end{Highlighting}
\end{Shaded}

\begin{Shaded}
\begin{Highlighting}[]
\DecValTok{2} \SpecialCharTok{+} \DecValTok{2} \CommentTok{\# add two numbers}
\end{Highlighting}
\end{Shaded}

\textbf{Note:} All question answers can be found at the end of the
lesson.

\end{tcolorbox}

A fantastic use of comments is to separate your scripts into sections.
If you put four dashes after a comment, RStudio will create a new
section in your code:

\begin{Shaded}
\begin{Highlighting}[]
\DocumentationTok{\#\# New section {-}{-}{-}{-}}
\end{Highlighting}
\end{Shaded}

This has two nice benefits. Firstly, you can click on the little arrow
beside the section header to fold, or collapse, that section of code:

\includegraphics[width=1.78125in,height=\textheight]{images/section_folder_icon.jpg}

Second, you can click on the ``Outline'' icon at the top right of the
Editor to view and navigate through all the contents in your script:

\includegraphics[width=2.02083in,height=\textheight]{images/rstudio_outline.jpg}

\hypertarget{r-s-a-calculator}{%
\section{R s a calculator}\label{r-s-a-calculator}}

R works as a calculator, and obeys the correct order of operations. Type
and run the following expressions and observe their output:

\begin{Shaded}
\begin{Highlighting}[]
\DecValTok{2} \SpecialCharTok{+} \DecValTok{2}
\end{Highlighting}
\end{Shaded}

\begin{verbatim}
[1] 4
\end{verbatim}

\begin{Shaded}
\begin{Highlighting}[]
\DecValTok{2} \SpecialCharTok{{-}} \DecValTok{2}
\end{Highlighting}
\end{Shaded}

\begin{verbatim}
[1] 0
\end{verbatim}

\begin{Shaded}
\begin{Highlighting}[]
\DecValTok{2} \SpecialCharTok{*} \DecValTok{2} \CommentTok{\# two times two }
\end{Highlighting}
\end{Shaded}

\begin{verbatim}
[1] 4
\end{verbatim}

\begin{Shaded}
\begin{Highlighting}[]
\DecValTok{2} \SpecialCharTok{/} \DecValTok{2} \CommentTok{\# two divided by two}
\end{Highlighting}
\end{Shaded}

\begin{verbatim}
[1] 1
\end{verbatim}

\begin{Shaded}
\begin{Highlighting}[]
\DecValTok{2} \SpecialCharTok{\^{}} \DecValTok{2} \CommentTok{\# two raised to the power of two}
\end{Highlighting}
\end{Shaded}

\begin{verbatim}
[1] 4
\end{verbatim}

\begin{Shaded}
\begin{Highlighting}[]
\DecValTok{2} \SpecialCharTok{+} \DecValTok{2} \SpecialCharTok{*} \DecValTok{2}   \CommentTok{\# this is evaluated following the order of operations}
\end{Highlighting}
\end{Shaded}

\begin{verbatim}
[1] 6
\end{verbatim}

\begin{Shaded}
\begin{Highlighting}[]
\FunctionTok{sqrt}\NormalTok{(}\DecValTok{100}\NormalTok{)   }\CommentTok{\# square root}
\end{Highlighting}
\end{Shaded}

\begin{verbatim}
[1] 10
\end{verbatim}

The square root command shown on the last line is a good example of an R
\emph{function}, where \texttt{100} is the \emph{argument} to the
function. You will see more functions soon.

\begin{tcolorbox}[enhanced jigsaw, colframe=quarto-callout-note-color-frame, rightrule=.15mm, opacityback=0, breakable, coltitle=black, colbacktitle=quarto-callout-note-color!10!white, bottomrule=.15mm, leftrule=.75mm, toprule=.15mm, arc=.35mm, bottomtitle=1mm, colback=white, left=2mm, opacitybacktitle=0.6, titlerule=0mm, title=\textcolor{quarto-callout-note-color}{\faInfo}\hspace{0.5em}{Reminder}, toptitle=1mm]

We hope you remember the shortcut to run code!

To \textbf{run a single line of code}, place your cursor anywhere on
that line, then hit \texttt{Command} + \texttt{Enter} on macOS, or
\texttt{Control} + \texttt{Enter} on Windows.

To \textbf{run multiple lines}, drag your cursor to highlight the
relevant lines then again press \texttt{Command}/\texttt{Control} +
\texttt{Enter}.

\end{tcolorbox}

\begin{tcolorbox}[enhanced jigsaw, colframe=quarto-callout-tip-color-frame, rightrule=.15mm, opacityback=0, breakable, coltitle=black, colbacktitle=quarto-callout-tip-color!10!white, bottomrule=.15mm, leftrule=.75mm, toprule=.15mm, arc=.35mm, bottomtitle=1mm, colback=white, left=2mm, opacitybacktitle=0.6, titlerule=0mm, title=\textcolor{quarto-callout-tip-color}{\faLightbulb}\hspace{0.5em}{Practice}, toptitle=1mm]

\textbf{Question 2}

In the following expression, which sign is evaluated first by R, the
minus or the division?

\begin{Shaded}
\begin{Highlighting}[]
\DecValTok{2} \SpecialCharTok{{-}} \DecValTok{2} \SpecialCharTok{/} \DecValTok{2}
\end{Highlighting}
\end{Shaded}

\begin{verbatim}
[1] 1
\end{verbatim}

\end{tcolorbox}

\hypertarget{formatting-code}{%
\section{Formatting code}\label{formatting-code}}

R does not care how you choose to space out your code.

For the math operations we did above, all the following would be valid
code:

\begin{Shaded}
\begin{Highlighting}[]
\DecValTok{2}\SpecialCharTok{+}\DecValTok{2}
\end{Highlighting}
\end{Shaded}

\begin{verbatim}
[1] 4
\end{verbatim}

\begin{Shaded}
\begin{Highlighting}[]
\DecValTok{2} \SpecialCharTok{+} \DecValTok{2}
\end{Highlighting}
\end{Shaded}

\begin{verbatim}
[1] 4
\end{verbatim}

\begin{Shaded}
\begin{Highlighting}[]
\DecValTok{2}                \SpecialCharTok{+}                   \DecValTok{2}
\end{Highlighting}
\end{Shaded}

\begin{verbatim}
[1] 4
\end{verbatim}

Similarly, for the \texttt{sqrt()} function used above, any of these
would be valid:

\begin{Shaded}
\begin{Highlighting}[]
\FunctionTok{sqrt}\NormalTok{(}\DecValTok{100}\NormalTok{)}
\end{Highlighting}
\end{Shaded}

\begin{verbatim}
[1] 10
\end{verbatim}

\begin{Shaded}
\begin{Highlighting}[]
\FunctionTok{sqrt}\NormalTok{(    }\DecValTok{100}\NormalTok{     )}
\end{Highlighting}
\end{Shaded}

\begin{verbatim}
[1] 10
\end{verbatim}

\begin{Shaded}
\begin{Highlighting}[]
\DocumentationTok{\#\# you can even space the command out over multiple lines}
\FunctionTok{sqrt}\NormalTok{(  }
  \DecValTok{100}
\NormalTok{    )}
\end{Highlighting}
\end{Shaded}

\begin{verbatim}
[1] 10
\end{verbatim}

But of course, you should try to space out your code in sensible ways.
What exactly is ``sensible''? Well, it may be hard for you to know at
the moment. Over time, as you read other people's code, you will learn
that there are certain R \emph{conventions} for code spacing and
formatting.

In the meantime, you can ask RStudio to help format your code for you.
To do this, highlight any section of code you want to reformat, and, on
the RStudio menu, go to \texttt{Code\ \textgreater{}\ Reformat\ Code},
or use the shortcut \texttt{Shift} + \texttt{Command/Control} +
\texttt{A}.

\begin{tcolorbox}[enhanced jigsaw, colframe=quarto-callout-caution-color-frame, rightrule=.15mm, opacityback=0, breakable, coltitle=black, colbacktitle=quarto-callout-caution-color!10!white, bottomrule=.15mm, leftrule=.75mm, toprule=.15mm, arc=.35mm, bottomtitle=1mm, colback=white, left=2mm, opacitybacktitle=0.6, titlerule=0mm, title=\textcolor{quarto-callout-caution-color}{\faFire}\hspace{0.5em}{Watch Out}, toptitle=1mm]

\textbf{Stuck on the + sign}

If you run an incomplete line of code, R will print a \texttt{+} sign to
indicate that it is waiting for you to finish the code.

For example, if you run the following code:

\begin{Shaded}
\begin{Highlighting}[]
\FunctionTok{sqrt}\NormalTok{(}\DecValTok{100}
\end{Highlighting}
\end{Shaded}

you will not get the output you expect (10). Rather the console will
\texttt{sqrt(} and a \texttt{+} sign:

\includegraphics[width=0.83333in,height=0.35417in]{images/sqrt_missing_parenth.jpg}

R is waiting for you complete the closing parenthesis. You can complete
the code and get rid of the \texttt{+} by just entering the missing
parenthesis:

\begin{Shaded}
\begin{Highlighting}[]
\ErrorTok{)}
\end{Highlighting}
\end{Shaded}

\includegraphics[width=1.09375in,height=0.51042in]{images/sqrt_completed_parenth.jpg}

Alternatively, press the escape key, \texttt{ESC} while your cursor is
in the console to start over.

\end{tcolorbox}

\hypertarget{objects-in-r}{%
\section{Objects in R}\label{objects-in-r}}

\hypertarget{create-an-object}{%
\subsection{Create an object}\label{create-an-object}}

When you run code as we have been doing above, the result of the command
(or its \emph{value}) is simply displayed in the console---it is not
stored anywhere.

\begin{Shaded}
\begin{Highlighting}[]
\DecValTok{2} \SpecialCharTok{+} \DecValTok{2} \CommentTok{\# R prints this result, 4, but does not store it }
\end{Highlighting}
\end{Shaded}

\begin{verbatim}
[1] 4
\end{verbatim}

To store a value for future use, assign it to an \emph{object} with the
\emph{assignment operator\textbf{,}} \texttt{\textless{}-} :

\begin{Shaded}
\begin{Highlighting}[]
\NormalTok{my\_obj }\OtherTok{\textless{}{-}} \DecValTok{2} \SpecialCharTok{+} \DecValTok{2} \CommentTok{\# assign the result of \textasciigrave{}2 + 2 \textasciigrave{} to the object called \textasciigrave{}my\_obj\textasciigrave{}}
\NormalTok{my\_obj }\CommentTok{\# print my\_obj}
\end{Highlighting}
\end{Shaded}

\begin{verbatim}
[1] 4
\end{verbatim}

The assignment operator, \texttt{\textless{}-} , is made of the `less
than' sign, \texttt{\textless{}} , and a minus, \texttt{-}. You will use
it thousands of times over your R lifetime, so please don't type it
manually! Instead, use RStudio's shortcut, \textbf{\texttt{alt}} +
\textbf{\texttt{-}} (\textbf{alt} AND \textbf{minus}) on Windows or
\textbf{\texttt{option}} + \textbf{\texttt{-}} (\textbf{option} AND
\textbf{minus}) on macOS.

\begin{tcolorbox}[enhanced jigsaw, colframe=quarto-callout-note-color-frame, rightrule=.15mm, opacityback=0, breakable, coltitle=black, colbacktitle=quarto-callout-note-color!10!white, bottomrule=.15mm, leftrule=.75mm, toprule=.15mm, arc=.35mm, bottomtitle=1mm, colback=white, left=2mm, opacitybacktitle=0.6, titlerule=0mm, title=\textcolor{quarto-callout-note-color}{\faInfo}\hspace{0.5em}{Side Note}, toptitle=1mm]

Also note that you can use the \emph{equals} sign, \texttt{=}, for
assignment.

\begin{Shaded}
\begin{Highlighting}[]
\NormalTok{my\_obj }\OtherTok{=} \DecValTok{2} \SpecialCharTok{+} \DecValTok{2} 
\end{Highlighting}
\end{Shaded}

But this is not commonly used by the R community (mostly for historical
reasons), so we discourage it too. Follow the convention and use
\texttt{\textless{}-}.

\end{tcolorbox}

Now that you've created the object \texttt{my\_obj}, R knows all about
it and will keep track of it during this R session. You can view any
created objects in the \emph{Environment} tab of RStudio.

\includegraphics[width=3.25in,height=\textheight]{images/rstudio_environment_object.png}

\hypertarget{what-is-an-object}{%
\subsection{What is an object?}\label{what-is-an-object}}

So what exactly is an object? Think of it as a named bucket that can
contain anything. When you run the code below:

\begin{Shaded}
\begin{Highlighting}[]
\NormalTok{my\_obj }\OtherTok{\textless{}{-}} \DecValTok{20}
\end{Highlighting}
\end{Shaded}

you are telling R, ``put the number 20 inside a bucket named
`my\_obj'\,''.

\includegraphics[width=1.625in,height=\textheight]{images/my_obj_20.jpg}

Once the code is run, we would say, in R terms, that ``the value of
object called \texttt{my\_obj} is 20''.

\begin{center}\rule{0.5\linewidth}{0.5pt}\end{center}

And if you run this code:

\begin{Shaded}
\begin{Highlighting}[]
\NormalTok{first\_name }\OtherTok{\textless{}{-}} \StringTok{"Joanna"}
\end{Highlighting}
\end{Shaded}

you are instructing R to ``put the value `Joanna' inside the bucket
called `first\_name'\,''.

\includegraphics[width=2.44792in,height=\textheight]{images/object_first_name_joanna.jpg}

Once the code is run, we would say, in R terms, that ``the value of the
\texttt{first\_name} object is Joanna''.

\begin{center}\rule{0.5\linewidth}{0.5pt}\end{center}

Note that R evaluates the code \emph{before} putting it inside the
bucket.

So, before when we ran this code,

\begin{Shaded}
\begin{Highlighting}[]
\NormalTok{my\_obj }\OtherTok{\textless{}{-}} \DecValTok{2} \SpecialCharTok{+} \DecValTok{2}
\end{Highlighting}
\end{Shaded}

R firsts does the calculation of \texttt{2\ +\ 2}, then stores the
result, 4, inside the object.

\includegraphics[width=2.32292in,height=\textheight]{images/object_two_plus_two.jpg}

\begin{tcolorbox}[enhanced jigsaw, colframe=quarto-callout-tip-color-frame, rightrule=.15mm, opacityback=0, breakable, coltitle=black, colbacktitle=quarto-callout-tip-color!10!white, bottomrule=.15mm, leftrule=.75mm, toprule=.15mm, arc=.35mm, bottomtitle=1mm, colback=white, left=2mm, opacitybacktitle=0.6, titlerule=0mm, title=\textcolor{quarto-callout-tip-color}{\faLightbulb}\hspace{0.5em}{Practice}, toptitle=1mm]

\textbf{Question 3}

Consider the code chunk below:

\begin{Shaded}
\begin{Highlighting}[]
\NormalTok{result }\OtherTok{\textless{}{-}}  \DecValTok{2} \SpecialCharTok{+} \DecValTok{2} \SpecialCharTok{+} \DecValTok{2}
\end{Highlighting}
\end{Shaded}

What is the value of the \texttt{result} object created?

A. \texttt{2\ +\ 2\ +\ 2}

B. numeric

C. 6

\end{tcolorbox}

\hypertarget{datasets-are-objects-too}{%
\subsection{Datasets are objects too}\label{datasets-are-objects-too}}

So far, you have been working with very simple objects. You may be
thinking ``Where are the spreadsheets and datasets? Why are we writing
\texttt{my\_obj\ \textless{}-\ 2\ +\ 2}? Is this a primary school maths
class?!''

Be patient.

We want you to get familiar with the concept of an R object because once
you start dealing with real datasets, these will also be stored as R
objects.

Let's see a preview of this now. Type out the code below to download a
dataset on Ebola cases that we stored on Google Drive and put it in the
object \texttt{ebola\_sierra\_leone\_data}.

\begin{Shaded}
\begin{Highlighting}[]
\NormalTok{ebola\_sierra\_leone\_data }\OtherTok{\textless{}{-}} \FunctionTok{read.csv}\NormalTok{(}\StringTok{"https://tinyurl.com/ebola{-}data{-}sample"}\NormalTok{)}
\NormalTok{ebola\_sierra\_leone\_data }\CommentTok{\# print ebola\_data}
\end{Highlighting}
\end{Shaded}

\begin{verbatim}
   id age sex    status date_of_onset date_of_sample district
1 167  55   M confirmed    2014-06-15     2014-06-21   Kenema
2 129  41   M confirmed    2014-06-13     2014-06-18 Kailahun
3 270  12   F confirmed    2014-06-28     2014-07-03 Kailahun
4 187  NA   F confirmed    2014-06-19     2014-06-24 Kailahun
5  85  20   M confirmed    2014-06-08     2014-06-24 Kailahun
\end{verbatim}

This data contains a sample of patient information from the 2014-2016
Ebola outbreak in Sierra Leone.

\begin{center}\rule{0.5\linewidth}{0.5pt}\end{center}

Because you can store datasets as objects, its very easy to work with
multiple datasets at the same time.

Below, we import and view another dataset from the web:

\begin{Shaded}
\begin{Highlighting}[]
\NormalTok{diabetes\_china }\OtherTok{\textless{}{-}} \FunctionTok{read.csv}\NormalTok{(}\StringTok{"https://tinyurl.com/diabetes{-}china"}\NormalTok{)}
\end{Highlighting}
\end{Shaded}

Because the dataset above is quite large, it may be helpful to look at
it in the data viewer:

\begin{Shaded}
\begin{Highlighting}[]
\FunctionTok{View}\NormalTok{(diabetes\_china)}
\end{Highlighting}
\end{Shaded}

Notice that both datasets now appear in your \emph{Environment} tab.

\begin{tcolorbox}[enhanced jigsaw, colframe=quarto-callout-note-color-frame, rightrule=.15mm, opacityback=0, breakable, coltitle=black, colbacktitle=quarto-callout-note-color!10!white, bottomrule=.15mm, leftrule=.75mm, toprule=.15mm, arc=.35mm, bottomtitle=1mm, colback=white, left=2mm, opacitybacktitle=0.6, titlerule=0mm, title=\textcolor{quarto-callout-note-color}{\faInfo}\hspace{0.5em}{Side Note}, toptitle=1mm]

Rather than reading data from an internet drive as we did above, it is
more likely that you will have the data on your computer, and you will
want to read it into R from your there. We will cover this in a future
lesson.

Later in the course, we will also show you how to store and read data
from a web service like Google Drive, which is nice for easy
portability.

\end{tcolorbox}

\hypertarget{rename-an-object}{%
\subsection{Rename an object}\label{rename-an-object}}

You sometimes want to rename an object. It is not possible to do this
directly.

To rename an object, you make a copy of the object with a new name, and
delete the original.

For example, maybe we decide that the name of the
\texttt{ebola\_sierra\_leone\_data} object is too long. To change it to
the shorter ``ebola\_data'' run:

\begin{Shaded}
\begin{Highlighting}[]
\NormalTok{ebola\_data }\OtherTok{\textless{}{-}}\NormalTok{ ebola\_sierra\_leone\_data}
\end{Highlighting}
\end{Shaded}

This has copied the contents from the
\texttt{ebola\_sierra\_leone\_data} \emph{bucket} to a new
\texttt{ebola\_data} \emph{bucket}.

You can now get rid of the old \texttt{ebola\_sierra\_leone\_data}
bucket with the \texttt{rm()} function, which stands for ``remove'':

\begin{Shaded}
\begin{Highlighting}[]
\FunctionTok{rm}\NormalTok{(ebola\_sierra\_leone\_data)}
\end{Highlighting}
\end{Shaded}

\hypertarget{overwrite-an-object}{%
\subsection{Overwrite an object}\label{overwrite-an-object}}

Overwriting an object is like changing the \emph{contents} of a
\emph{bucket}.

For example, previously we ran this code to store the value ``Joanna''
inside the \texttt{first\_name} object:

\begin{Shaded}
\begin{Highlighting}[]
\NormalTok{first\_name }\OtherTok{\textless{}{-}} \StringTok{"Joanna"}
\end{Highlighting}
\end{Shaded}

To change this to a different, simply re-run the line with a different
value:

\begin{Shaded}
\begin{Highlighting}[]
\NormalTok{first\_name }\OtherTok{\textless{}{-}} \StringTok{"Luigi"}
\end{Highlighting}
\end{Shaded}

You can take a look at the Environment tab to observe the change.

\hypertarget{working-with-objects}{%
\subsection{Working with objects}\label{working-with-objects}}

Most of your time in R will be spent manipulating R objects. Let's see
some quick examples.

You can run simple commands on objects. For example, below we store the
value \texttt{100} in an object and then take the square root of the
object:

\begin{Shaded}
\begin{Highlighting}[]
\NormalTok{my\_number }\OtherTok{\textless{}{-}} \DecValTok{100}
\FunctionTok{sqrt}\NormalTok{(my\_number)}
\end{Highlighting}
\end{Shaded}

\begin{verbatim}
[1] 10
\end{verbatim}

R ``sees'' \texttt{my\_number} as the number 100, and so is able to
evaluate it's square root.

\begin{center}\rule{0.5\linewidth}{0.5pt}\end{center}

You can also combine existing objects to create new objects. For
example, type out the code below to add \texttt{my\_number} to itself,
and store the result in a new object called \texttt{my\_sum}:

\begin{Shaded}
\begin{Highlighting}[]
\NormalTok{my\_sum }\OtherTok{\textless{}{-}}\NormalTok{ my\_number }\SpecialCharTok{+}\NormalTok{ my\_number}
\end{Highlighting}
\end{Shaded}

What should be the value of \texttt{my\_sum}? First take a guess, then
check it.

\begin{tcolorbox}[enhanced jigsaw, colframe=quarto-callout-note-color-frame, rightrule=.15mm, opacityback=0, breakable, coltitle=black, colbacktitle=quarto-callout-note-color!10!white, bottomrule=.15mm, leftrule=.75mm, toprule=.15mm, arc=.35mm, bottomtitle=1mm, colback=white, left=2mm, opacitybacktitle=0.6, titlerule=0mm, title=\textcolor{quarto-callout-note-color}{\faInfo}\hspace{0.5em}{Side Note}, toptitle=1mm]

To check the value of an object, such as \texttt{my\_sum}, you can type
and run just the code \texttt{my\_sum} in the Console or the Editor.
Alternatively, you can simply highlight the value \texttt{my\_sum} in
the existing code and press \texttt{Command/Control} + \texttt{Enter}.

\end{tcolorbox}

\begin{center}\rule{0.5\linewidth}{0.5pt}\end{center}

But of course, most of your analysis will involve working with
\emph{data} objects, such as the \texttt{ebola\_data} object we created
previously.

Let's see a very simple example of how to interact with a data object;
we will tackle it properly in the next lesson.

To get a table of the different sex distribution of patients in the
\texttt{ebola\_data} object, we can run the following:

\begin{Shaded}
\begin{Highlighting}[]
\FunctionTok{table}\NormalTok{(ebola\_data}\SpecialCharTok{$}\NormalTok{sex)}
\end{Highlighting}
\end{Shaded}

\begin{verbatim}

  F   M 
124  76 
\end{verbatim}

The dollar sign symbol, \texttt{\$}, above allowed us subset to a
specific column.

\begin{tcolorbox}[enhanced jigsaw, colframe=quarto-callout-tip-color-frame, rightrule=.15mm, opacityback=0, breakable, coltitle=black, colbacktitle=quarto-callout-tip-color!10!white, bottomrule=.15mm, leftrule=.75mm, toprule=.15mm, arc=.35mm, bottomtitle=1mm, colback=white, left=2mm, opacitybacktitle=0.6, titlerule=0mm, title=\textcolor{quarto-callout-tip-color}{\faLightbulb}\hspace{0.5em}{Practice}, toptitle=1mm]

\textbf{Question 4}

\begin{enumerate}
\def\labelenumi{\alph{enumi}.}
\tightlist
\item
  Consider the code below. What is the value of the \texttt{answer}
  object?
\end{enumerate}

\begin{Shaded}
\begin{Highlighting}[]
\NormalTok{eight }\OtherTok{\textless{}{-}} \DecValTok{9}
\NormalTok{answer }\OtherTok{\textless{}{-}}\NormalTok{ eight }\SpecialCharTok{{-}} \DecValTok{8}
\end{Highlighting}
\end{Shaded}

\begin{enumerate}
\def\labelenumi{\alph{enumi}.}
\setcounter{enumi}{1}
\tightlist
\item
  Use \texttt{table()} to make a table with the distribution of patients
  across districts in the \texttt{ebola\_data} object.
\end{enumerate}

\end{tcolorbox}

\hypertarget{some-errors-with-objects}{%
\subsection{Some errors with objects}\label{some-errors-with-objects}}

\begin{Shaded}
\begin{Highlighting}[]
\NormalTok{first\_name }\OtherTok{\textless{}{-}} \StringTok{"Luigi"}
\NormalTok{last\_name }\OtherTok{\textless{}{-}} \StringTok{"Fenway"}
\end{Highlighting}
\end{Shaded}

\begin{Shaded}
\begin{Highlighting}[]
\NormalTok{full\_name }\OtherTok{\textless{}{-}}\NormalTok{ first\_name }\SpecialCharTok{+}\NormalTok{ last\_name}
\end{Highlighting}
\end{Shaded}

\begin{verbatim}
Error in first_name + last_name : non-numeric argument to binary operator
\end{verbatim}

The error message tells you that these objects are not numbers and
therefore cannot be added with \texttt{+}. This is a fairly common error
type, caused by trying to do inappropriate things to your objects. Be
careful about this.

In this particular case, we can use the function \texttt{paste()} to put
these two objects together:

\begin{Shaded}
\begin{Highlighting}[]
\NormalTok{full\_name }\OtherTok{\textless{}{-}} \FunctionTok{paste}\NormalTok{(first\_name, last\_name)}
\NormalTok{full\_name}
\end{Highlighting}
\end{Shaded}

\begin{verbatim}
[1] "Luigi Fenway"
\end{verbatim}

\begin{center}\rule{0.5\linewidth}{0.5pt}\end{center}

Another error you'll get a lot is
\texttt{Error:\ object\ \textquotesingle{}XXX\textquotesingle{}\ not\ found}.
For example:

\begin{Shaded}
\begin{Highlighting}[]
\NormalTok{my\_number }\OtherTok{\textless{}{-}} \DecValTok{48} \CommentTok{\# define \textasciigrave{}my\_obj\textasciigrave{}}
\NormalTok{My\_number }\SpecialCharTok{+} \DecValTok{2} \CommentTok{\# attempt to add 2 to \textasciigrave{}my\_obj\textasciigrave{}}
\end{Highlighting}
\end{Shaded}

\begin{verbatim}
Error: object 'My_number' not found
\end{verbatim}

Here, R returns an error message because we haven't created (or
\emph{defined}) the object \texttt{My\_obj} yet. (Recall that R is
case-sensitive.)

\begin{center}\rule{0.5\linewidth}{0.5pt}\end{center}

When you first start learning R, dealing with errors can be frustrating.
They're often difficult to understand (e.g.~what exactly does
``\emph{non-numeric argument to binary operator}'' mean?).

Try Googling any error messages you get and browsing through the first
few results. This will lead you to forums (e.g.~stackoverflow.com) where
other R learners have complained about the same error. Here you may find
explanations of, and solutions to, your problems.

\begin{tcolorbox}[enhanced jigsaw, colframe=quarto-callout-tip-color-frame, rightrule=.15mm, opacityback=0, breakable, coltitle=black, colbacktitle=quarto-callout-tip-color!10!white, bottomrule=.15mm, leftrule=.75mm, toprule=.15mm, arc=.35mm, bottomtitle=1mm, colback=white, left=2mm, opacitybacktitle=0.6, titlerule=0mm, title=\textcolor{quarto-callout-tip-color}{\faLightbulb}\hspace{0.5em}{Practice}, toptitle=1mm]

\textbf{Question 5}

\begin{enumerate}
\def\labelenumi{\alph{enumi}.}
\tightlist
\item
  The code below returns an error. Why?
\end{enumerate}

\begin{Shaded}
\begin{Highlighting}[]
\NormalTok{my\_first\_name }\OtherTok{\textless{}{-}} \StringTok{"Kene"}
\NormalTok{my\_last\_name }\OtherTok{\textless{}{-}} \StringTok{"Nwosu"}
\NormalTok{my\_first\_name }\SpecialCharTok{+}\NormalTok{ my\_last\_name}
\end{Highlighting}
\end{Shaded}

\begin{enumerate}
\def\labelenumi{\alph{enumi}.}
\setcounter{enumi}{1}
\tightlist
\item
  The code below returns an error. Why? (Look carefully)
\end{enumerate}

\begin{Shaded}
\begin{Highlighting}[]
\NormalTok{my\_1st\_name }\OtherTok{\textless{}{-}} \StringTok{"Kene"}
\NormalTok{my\_last\_name }\OtherTok{\textless{}{-}} \StringTok{"Nwosu"}

\FunctionTok{paste}\NormalTok{(my\_Ist\_name, my\_last\_name)}
\end{Highlighting}
\end{Shaded}

\end{tcolorbox}

\hypertarget{naming-objects}{%
\subsection{Naming objects}\label{naming-objects}}

\begin{quote}
There are only \textbf{\emph{two hard things}} in Computer Science:
cache invalidation and \textbf{\emph{naming things}}.

--- Phil Karlton.
\end{quote}

Because much of your work in R involves interacting with objects you
have created, picking intelligent names for these objects is important.

Naming objects is difficult because names should be both \textbf{short}
(so that you can type them quickly) and \textbf{informative} (so that
you can easily remember what is inside the object), and these two goals
are often in conflict.

So names that are too long, like the one below, are bad because they
take forever to type.

\begin{Shaded}
\begin{Highlighting}[]
\NormalTok{sample\_of\_the\_ebola\_outbreak\_dataset\_from\_sierra\_leone\_in\_2014}
\end{Highlighting}
\end{Shaded}

And a name like \texttt{data} is bad because it is not informative; the
name does not give a good idea of what the object is.

As you write more R code, you will learn how to write short and
informative names.

\begin{center}\rule{0.5\linewidth}{0.5pt}\end{center}

For names with multiple words, there are a few conventions for how to
separate the words:

\begin{Shaded}
\begin{Highlighting}[]
\NormalTok{snake\_case }\OtherTok{\textless{}{-}} \StringTok{"Snake case uses underscores"}
\NormalTok{period.case }\OtherTok{\textless{}{-}} \StringTok{"Period case uses periods"}
\NormalTok{camelCase }\OtherTok{\textless{}{-}} \StringTok{"Camel case capitalizes new words (but not the first word)"}
\end{Highlighting}
\end{Shaded}

We recommend snake\_case, which uses all lower-case words, and separates
words with \texttt{\_}.

\begin{center}\rule{0.5\linewidth}{0.5pt}\end{center}

Note too that there are some limitations on objects' names:

\begin{itemize}
\item
  names must start with a letter. So \texttt{2014\_data} is not a valid
  name (because it starts with a number).
\item
  names can only contain letters, numbers, periods (\texttt{.}) and
  underscores (\texttt{\_}). So \texttt{ebola-data} or
  \texttt{ebola\textasciitilde{}data} or \texttt{ebola\ data} with a
  space are not valid names.
\end{itemize}

If you really want to use these characters in your object names, you can
enclose the names in backticks:

\begin{verbatim}
`ebola-data`
`ebola~data`
`ebola data`
\end{verbatim}

All of the above are valid R object names. For example, type and run the
following code:

\begin{Shaded}
\begin{Highlighting}[]
\StringTok{\textasciigrave{}}\AttributeTok{ebola\textasciitilde{}data}\StringTok{\textasciigrave{}} \OtherTok{\textless{}{-}}\NormalTok{ ebola\_data}
\StringTok{\textasciigrave{}}\AttributeTok{ebola\textasciitilde{}data}\StringTok{\textasciigrave{}}
\end{Highlighting}
\end{Shaded}

But in general you should avoid using backticks to rescue bad object
names. Just write proper names.

\begin{tcolorbox}[enhanced jigsaw, colframe=quarto-callout-tip-color-frame, rightrule=.15mm, opacityback=0, breakable, coltitle=black, colbacktitle=quarto-callout-tip-color!10!white, bottomrule=.15mm, leftrule=.75mm, toprule=.15mm, arc=.35mm, bottomtitle=1mm, colback=white, left=2mm, opacitybacktitle=0.6, titlerule=0mm, title=\textcolor{quarto-callout-tip-color}{\faLightbulb}\hspace{0.5em}{Practice}, toptitle=1mm]

\textbf{Question 6}

In the code chunk below, we are attempting to take the top 20 rows of
the \texttt{ebola\_data} table. All but one of these lines has an error.
Which line will run properly?

\begin{Shaded}
\begin{Highlighting}[]
\NormalTok{20\_top\_rows }\OtherTok{\textless{}{-}} \FunctionTok{head}\NormalTok{(ebola\_data, }\DecValTok{20}\NormalTok{)}
\NormalTok{twenty}\SpecialCharTok{{-}}\NormalTok{top}\SpecialCharTok{{-}}\NormalTok{rows }\OtherTok{\textless{}{-}} \FunctionTok{head}\NormalTok{(ebola\_data, }\DecValTok{20}\NormalTok{)}
\NormalTok{top\_20\_rows }\OtherTok{\textless{}{-}} \FunctionTok{head}\NormalTok{(ebola\_data, }\DecValTok{20}\NormalTok{)}
\end{Highlighting}
\end{Shaded}

\end{tcolorbox}

\hypertarget{functions}{%
\section{Functions}\label{functions}}

Much of your work in R will involve calling \emph{functions}.

You can think of each function as a machine that takes in some input (or
\emph{arguments}) and returns some output.

\includegraphics[width=6.35417in,height=\textheight]{images/apple_slicing_function_analogy.png}

So far you have already seen many functions, including, \texttt{sqrt()},
\texttt{paste()} and \texttt{plot()}. Run the lines below to refresh
your memory:

\begin{Shaded}
\begin{Highlighting}[]
\FunctionTok{sqrt}\NormalTok{(}\DecValTok{100}\NormalTok{)}
\FunctionTok{paste}\NormalTok{(}\StringTok{"I am number"}\NormalTok{, }\DecValTok{2} \SpecialCharTok{+} \DecValTok{2}\NormalTok{)}
\FunctionTok{plot}\NormalTok{(women)}
\end{Highlighting}
\end{Shaded}

\hypertarget{basic-function-syntax}{%
\subsection{Basic function syntax}\label{basic-function-syntax}}

The standard way to call a function is to provide a \emph{value} for
each \emph{argument}:

\begin{Shaded}
\begin{Highlighting}[]
\FunctionTok{function\_name}\NormalTok{(}\AttributeTok{argument1 =} \StringTok{"value"}\NormalTok{, }\AttributeTok{argument2 =} \StringTok{"value"}\NormalTok{)}
\end{Highlighting}
\end{Shaded}

Let's demonstrate this with the \texttt{head()} function, which returns
the first few elements of an object.

To return the first three rows of the Ebola dataset, you run:

\begin{Shaded}
\begin{Highlighting}[]
\FunctionTok{head}\NormalTok{(}\AttributeTok{x =}\NormalTok{ ebola\_data, }\AttributeTok{n =} \DecValTok{3}\NormalTok{)}
\end{Highlighting}
\end{Shaded}

\begin{verbatim}
   id age sex    status date_of_onset date_of_sample district
1 167  55   M confirmed    2014-06-15     2014-06-21   Kenema
2 129  41   M confirmed    2014-06-13     2014-06-18 Kailahun
3 270  12   F confirmed    2014-06-28     2014-07-03 Kailahun
\end{verbatim}

In the code above, \texttt{head()} takes in two arguments:

\begin{itemize}
\item
  \texttt{x} , the object of interest, and
\item
  \texttt{n}, the number of elements to return.
\end{itemize}

We can also swap the order of the arguments:

\begin{Shaded}
\begin{Highlighting}[]
\FunctionTok{head}\NormalTok{(}\AttributeTok{n =} \DecValTok{3}\NormalTok{, }\AttributeTok{x =}\NormalTok{ ebola\_data)}
\end{Highlighting}
\end{Shaded}

\begin{verbatim}
   id age sex    status date_of_onset date_of_sample district
1 167  55   M confirmed    2014-06-15     2014-06-21   Kenema
2 129  41   M confirmed    2014-06-13     2014-06-18 Kailahun
3 270  12   F confirmed    2014-06-28     2014-07-03 Kailahun
\end{verbatim}

\begin{center}\rule{0.5\linewidth}{0.5pt}\end{center}

If you put the argument values in the right order, you can skip typing
their names. So the following two lines of code are equivalent and both
run:

\begin{Shaded}
\begin{Highlighting}[]
\FunctionTok{head}\NormalTok{(}\AttributeTok{x =}\NormalTok{ ebola\_data, }\AttributeTok{n =} \DecValTok{3}\NormalTok{)}
\end{Highlighting}
\end{Shaded}

\begin{verbatim}
   id age sex    status date_of_onset date_of_sample district
1 167  55   M confirmed    2014-06-15     2014-06-21   Kenema
2 129  41   M confirmed    2014-06-13     2014-06-18 Kailahun
3 270  12   F confirmed    2014-06-28     2014-07-03 Kailahun
\end{verbatim}

\begin{Shaded}
\begin{Highlighting}[]
\FunctionTok{head}\NormalTok{(ebola\_data, }\DecValTok{3}\NormalTok{)}
\end{Highlighting}
\end{Shaded}

\begin{verbatim}
   id age sex    status date_of_onset date_of_sample district
1 167  55   M confirmed    2014-06-15     2014-06-21   Kenema
2 129  41   M confirmed    2014-06-13     2014-06-18 Kailahun
3 270  12   F confirmed    2014-06-28     2014-07-03 Kailahun
\end{verbatim}

But if the argument values are in the wrong order, you will get an error
if you do not type the argument names. Below, the first line runs but
the second does not run:

\begin{Shaded}
\begin{Highlighting}[]
\FunctionTok{head}\NormalTok{(}\AttributeTok{n =} \DecValTok{3}\NormalTok{, }\AttributeTok{x =}\NormalTok{ ebola\_data)}
\FunctionTok{head}\NormalTok{(}\DecValTok{3}\NormalTok{, ebola\_data)}
\end{Highlighting}
\end{Shaded}

(To see the ``correct order'' for the arguments, take a look at the help
file for the \texttt{head()} function)

\begin{center}\rule{0.5\linewidth}{0.5pt}\end{center}

Some function arguments can be skipped altogether, because they have
\emph{default} values.

For example, with \texttt{head()}, the default value of \texttt{n} is 6,
so running just \texttt{head(ebola\_data)} will return the first 6 rows.

\begin{Shaded}
\begin{Highlighting}[]
\FunctionTok{head}\NormalTok{(ebola\_data)}
\end{Highlighting}
\end{Shaded}

\begin{verbatim}
   id age sex    status date_of_onset date_of_sample district
1 167  55   M confirmed    2014-06-15     2014-06-21   Kenema
2 129  41   M confirmed    2014-06-13     2014-06-18 Kailahun
3 270  12   F confirmed    2014-06-28     2014-07-03 Kailahun
4 187  NA   F confirmed    2014-06-19     2014-06-24 Kailahun
5  85  20   M confirmed    2014-06-08     2014-06-24 Kailahun
6 277  30   F confirmed    2014-06-29     2014-07-01   Kenema
\end{verbatim}

To see the arguments to a function, press the \textbf{Tab} key when your
cursor is inside the function's parentheses:

\includegraphics{images/head_definition_help.png}

\begin{tcolorbox}[enhanced jigsaw, colframe=quarto-callout-tip-color-frame, rightrule=.15mm, opacityback=0, breakable, coltitle=black, colbacktitle=quarto-callout-tip-color!10!white, bottomrule=.15mm, leftrule=.75mm, toprule=.15mm, arc=.35mm, bottomtitle=1mm, colback=white, left=2mm, opacitybacktitle=0.6, titlerule=0mm, title=\textcolor{quarto-callout-tip-color}{\faLightbulb}\hspace{0.5em}{Practice}, toptitle=1mm]

\textbf{Question 7}

In the code lines below, we are attempting to take the top 6 rows of the
\texttt{women} dataset (which is built into R). Which line is invalid?

\begin{Shaded}
\begin{Highlighting}[]
\FunctionTok{head}\NormalTok{(women)}
\FunctionTok{head}\NormalTok{(women, }\DecValTok{6}\NormalTok{)}
\FunctionTok{head}\NormalTok{(}\AttributeTok{x =}\NormalTok{ women, }\DecValTok{6}\NormalTok{)}
\FunctionTok{head}\NormalTok{(}\AttributeTok{x =}\NormalTok{ women, }\AttributeTok{n =} \DecValTok{6}\NormalTok{)}
\FunctionTok{head}\NormalTok{(}\DecValTok{6}\NormalTok{, women)}
\end{Highlighting}
\end{Shaded}

(If you are not sure, just try typing and running each line. Remember
that the goal here is for you to gain some practice.)

\end{tcolorbox}

\begin{center}\rule{0.5\linewidth}{0.5pt}\end{center}

Let's spend some time playing with another function, the
\texttt{paste()} function, which we already saw above, This function is
a bit special because it can take in any number of input arguments.

So you could have two arguments:

\begin{Shaded}
\begin{Highlighting}[]
\FunctionTok{paste}\NormalTok{(}\StringTok{"Luigi"}\NormalTok{, }\StringTok{"Fenway"}\NormalTok{)}
\end{Highlighting}
\end{Shaded}

\begin{verbatim}
[1] "Luigi Fenway"
\end{verbatim}

Or four arguments:

\begin{Shaded}
\begin{Highlighting}[]
\FunctionTok{paste}\NormalTok{(}\StringTok{"Luigi"}\NormalTok{, }\StringTok{"Fenway"}\NormalTok{, }\StringTok{"Luigi"}\NormalTok{, }\StringTok{"Fenway"}\NormalTok{)}
\end{Highlighting}
\end{Shaded}

\begin{verbatim}
[1] "Luigi Fenway Luigi Fenway"
\end{verbatim}

And so on up to infinity.

And as you might recall, we can also \texttt{paste()} named objects:

\begin{Shaded}
\begin{Highlighting}[]
\NormalTok{first\_name }\OtherTok{\textless{}{-}} \StringTok{"Luigi"}
\FunctionTok{paste}\NormalTok{(}\StringTok{"My name is"}\NormalTok{, first\_name, }\StringTok{"and my last name is"}\NormalTok{, last\_name)}
\end{Highlighting}
\end{Shaded}

\begin{verbatim}
[1] "My name is Luigi and my last name is Fenway"
\end{verbatim}

\begin{tcolorbox}[enhanced jigsaw, colframe=quarto-callout-note-color-frame, rightrule=.15mm, opacityback=0, breakable, coltitle=black, colbacktitle=quarto-callout-note-color!10!white, bottomrule=.15mm, leftrule=.75mm, toprule=.15mm, arc=.35mm, bottomtitle=1mm, colback=white, left=2mm, opacitybacktitle=0.6, titlerule=0mm, title=\textcolor{quarto-callout-note-color}{\faInfo}\hspace{0.5em}{Pro Tip}, toptitle=1mm]

Functions like \texttt{paste()} can take in many values because they
have a special argument, an ellipsis: \texttt{b